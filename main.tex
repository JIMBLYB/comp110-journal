\documentclass{article}
\usepackage[utf8]{inputenc}

\title{A Research Journal on 'A Method for Obtaining Digital Signatures and Public-Key Cryptosystems}
\author{JB230725}

\usepackage{natbib}
\usepackage{graphicx}

\begin{document}

\maketitle

\section{Introduction}

This is a research journal project about the context and influence of my chosen article \textit{A Method for Obtaining Digital Signatures and Public-Key Cryptosystems}, by \textit{R. L. Rivest, A. Shamir,} and \textit{L. Adleman} \cite{rivest1978_rsa} as part of my COMP 110 module at Falmouth University.

\section{What the Article Explains}

\textit{A Method for Obtaining Digital Signatures and Public-Key Cryptosystems} explains one of the possible methods to use the \textit{trap-door functions} outlined by W. Diffie and M. Hellman in their paper, \textit{New Directions in Cryptography}\cite{diffie1976new}. They found a method to create a private messaging system over a public channel which used what they called a \textit{trap-door function}; this worked by relying on the fact that there exist some calculations which can be easily calculated one way, but take an unreasonably long time to calculate backwards; The method suggested by Rivest, Shamir, and Adleman uses the difficulty in factorising a number that is a product of large prime numbers. 

Their article talks through why their method works, and how to do it, as well as how to crack their encryption and why each method of doing so is impractical to the point that they won't be attempted.

\section{Context}


\section{Conclusion}

See LearningSpace for the assignment brief, containing information on marking criteria and further guidance.

\bibliographystyle{plain}
\bibliography{references}
\end{document}