\documentclass{article}
\usepackage[utf8]{inputenc}

\title{A Research Journal on 'A Method for Obtaining Digital Signatures and Public-Key Cryptosystems}
\author{JB230725}

\usepackage{natbib}
\usepackage{graphicx}

\begin{document}

\maketitle

\section{Introduction}

This is a research journal project about the context and influence of my chosen article \textit{A Method for Obtaining Digital Signatures and Public-Key Cryptosystems}, by \textit{R. L. Rivest, A. Shamir,} and \textit{L. Adleman} \cite{rivest1978_rsa} as part of my COMP 110 module at Falmouth University.

\section{What the Article Explains}

\textit{A Method for Obtaining Digital Signatures and Public-Key Cryptosystems} explains one of the possible methods to use the \textit{trap-door functions} outlined by W. Diffie and M. Hellman in their paper, \textit{New Directions in Cryptography}\cite{diffie1976new}. They describe a \textit{trap-door function}, which can be easily calculated one way, but take an unreasonably long time to calculate backwards, and it's significance to cryptography; The method suggested by Rivest, Shamir, and Adleman (hereby referred to in this paper as \textbf{RSA Encryption}) uses the difficulty in factorising a number that is a product of large prime numbers. \newline
Their article talks through why RSA Encryption works, and how to do it, as well as how to crack their encryption and why each method of doing so is impractical to the point that they won't be attempted.

\section{Context}

The trap-door functions that Diffie and Hellman outlined in their paper\cite{diffie1976new} had been recognised as a possible solution to online security problems with the new digital systems being developed at the time, and they discussed the insecurities in the NBS Data Encryption Standard used at the time in their paper one year later\cite{diffie1977special}, so the need to find a better method was getting increasingly larger.\newline Furthermore, since online messaging was slowly being developed, Rivest, Shamir and Adleman noticed that for digital mail to truly overtake paper mail then there needed to be a way to prove who a message was sent by and that it was un-tampered-with; something that they called a 'digital signature'. They found a way to do this in a paper by R. Merkle: \textit{Secure Communications Over Insecure Channels} which described a method to allow private messaging over public messaging channels\cite{merkle1978secure}.\newline There were several existing methods of encryption using different trap-door functions, such as the method described by Hellman and Pohlig in their paper \textit{An improved algorithm for computing logarithms over GF (p) and its cryptographic significance (Corresp.)}\cite{pohlig1978improved} using Logarithms, and Levine and Brawley's \textit{Some cryptographic applications of permutation polynomials}\cite{levine1977some}, which suggests permutation polynomials over a Galois field, and that they are already pretty close as the permutations of a ring is the basis of a cypher alphabet.\newline The method described for RSA Encryption however, had a property that allowed it to be used to 'sign' a document, making it possible to prove who sent the message, and also impossible to deny having sent a message, it also makes it tamper-proof; this property allowed for messages of official significance to be sent digitally rather than by paper mail.

\section{Conclusion}

See LearningSpace for the assignment brief, containing information on marking criteria and further guidance.

\bibliographystyle{plain}
\bibliography{references}
\end{document}